\documentclass{article}
\usepackage{ifthen}
\usepackage{amssymb}
\usepackage{multicol}
\usepackage{graphicx}
\usepackage[absolute]{textpos}
\usepackage{amsmath, amscd, amssymb, amsthm, latexsym}
\usepackage{xspace,rotating,dsfont,ifthen}
\usepackage[spanish,activeacute]{babel}
\usepackage[utf8]{inputenc}
\usepackage{pgfpages}
\usepackage{pgf,pgfarrows,pgfnodes,pgfautomata,pgfheaps,xspace,dsfont}
\usepackage{listings}
\usepackage{multicol}
\usepackage{todonotes}
\usepackage{url}
\usepackage{float}
\usepackage{framed,mdframed}
\usepackage{cancel}

\usepackage[strict]{changepage}


\makeatletter


\newcommand\hfrac[2]{\genfrac{}{}{0pt}{}{#1}{#2}} %\hfrac{}{} es un \frac sin la linea del medio

\newcommand\Wider[2][3em]{% \Wider[3em]{} reduce los m\'argenes
\makebox[\linewidth][c]{%
  \begin{minipage}{\dimexpr\textwidth+#1\relax}
  \raggedright#2
  \end{minipage}%
  }%
}


\@ifclassloaded{beamer}{%
  \newcommand{\tocarEspacios}{%
    \addtolength{\leftskip}{4em}%
    \addtolength{\parindent}{-3em}%
  }%
}
{%
  \usepackage[top=1cm,bottom=2cm,left=1cm,right=1cm]{geometry}%
  \usepackage{color}%
  \newcommand{\tocarEspacios}{%
    \addtolength{\leftskip}{3em}%
    \setlength{\parindent}{0em}%
  }%
}

\usepackage{caratula}
\usepackage{enumerate}
\usepackage{hyperref}
\usepackage{graphicx}
\usepackage{amsfonts}
\usepackage{enumitem}
\usepackage{amsmath}

\decimalpoint
\hypersetup{colorlinks=true, linkcolor=black, urlcolor=blue}
\setlength{\parindent}{0em}
\setlength{\parskip}{0.5em}
\setcounter{tocdepth}{3} % profundidad de indice
\setcounter{section}{0} % nro de section
\renewcommand{\thesubsubsection}{\thesubsection.\Alph{subsubsection}}
\graphicspath{ {images/} }

% End latex config

\begin{document}

\titulo{Final 21/10/2021}
\fecha{2do cuatrimestre 2021}
\materia{Álgebra I}
\integrante{Yago Pajariño}{546/21}{ypajarino@dc.uba.ar}

%Carátula
\maketitle
\newpage

%Indice
\tableofcontents
\newpage

% Aca empieza lo propio del documento
\section{Final 21/10/2021}

\subsection{Ejercicio 1}

Por enunciado, se define la relación $R$ en $G_{50}$ como,
\begin{align*}
    zRw \iff zw^{24} \in G_2
\end{align*}
Por definición de raíces de la unidad, se que los elementos de $G_{50}$ son aquellos $ \alpha \in \mathbb{C}: \alpha^{50} = 1 $

Por lo tanto, es el conjunto de los elementos donde se define la relación, luego sean $ \alpha, \beta \in G_{50} $ la relación se define como:
\begin{align*}
    \alpha R\beta &\iff \alpha \beta^{24} \in G_2 \\
    &\iff (\alpha \beta^{24})^2 = 1 \\
    &\iff \begin{cases}
        \alpha \beta^{24} = 1 \\
        \alpha \beta^{24} = -1 \\
    \end{cases} \\
\end{align*}

\subsubsection{Pregunta i}
Busco probar que $R$ es una relación de equivalencia. Por definición, lo es si $R$ es reflexiva, transitiva y simétrica. Pruebo cada uno por separado.

\textbf{Reflexividad}

Por definición de reflexividad, $R$ es reflexiva $ \iff \forall a \in G_{50}: aRa $

Por definición de $R$,
\begin{align*}
    aRa &\iff a.a^{24} \in G_2 \\
    &\iff (a.a^{24})^2 = 1 \\
    &\iff (a^{25})^2 = 1 \\
    &\iff a^50 = 1 \\
\end{align*}
Y dado que $ a \in G_{50} \implies a^{50} = 1 $. Luego $R$ es reflexiva.

\textbf{Simetría}

Por definición de simetría, $R$ es simétrica $ \iff \forall \alpha, \beta \in G_{50}: \alpha R \beta \implies \beta R \alpha $

Luego, por definición de $R$,
\begin{align*}
    \alpha R \beta &\iff (\alpha \beta^{24})^2 = 1 \\
    &\iff \alpha^2 \beta^{48} = 1 \\
    &\iff (\alpha^2 \beta^{48})^{-1} = 1^{-1} \\
    &\iff \alpha^{-2} \beta^{-48} = 1 \\
    &\iff \alpha^{48} \beta^{2} = 1 \\
    &\iff (\alpha^{24} \beta)^2 = 1 \\
    &\iff \beta R \alpha \\
\end{align*}
Luego $R$ es simétrica.

\textbf{Transitividad}

Por definición, $R$ es transitiva $ \iff \forall a,b,c \in G_{50}: (aRb \wedge bRc) \implies aRc  $

Luego por definición de la relación se que,
\begin{align*}
    aRb &\iff ab^{24} \in g_2 \\
    bRc &\iff bc^{24} \in g_2
\end{align*}
Y quiero ver que $ ac^{24} \in G_2 $

Pero,
\begin{align*}
    (ab^{24})^2 = 1 \wedge (bc^{24})^2 = 1 &\implies (ab^{24})^2 \cdot (bc^{24})^2 = 1 \\
    &\iff (ab^{24}  \cdot bc^{24})^2 = 1 \\
    &\iff ab^{24} \cdot bc^{24} \in G_2 \\
\end{align*}
Luego,
\begin{align*}
    ab^{24} \cdot bc^{24} \in G_2 &\iff (ab^{24} \cdot bc^{24})^2 = 1 \\
    &\iff (ab^{25}c^{24})^2 = 1 \\
    &\iff (b^{25})^2 \cdot (ac^{24})^2 = 1 \\
    &\iff 1 \cdot (ac^{24})^2 = 1 \\
    &\iff (ac^{24})^2 = 1 \\
    &\iff aRc \\
\end{align*}
Como se quería probar, luego $R$ es transitiva.

Probado que $R$ es reflexiva, simétrica y transitiva; $R$ es una relación de equivalencia.

\subsection{Ejercicio 2}

Demostración usando el principio de inducción.

Defino $ p(n): F_n = \frac{L_{n-1} + L_{n+1}}{5}; \forall n \geq 1 $

\textbf{Caso base n = 1, n = 2 }

Por definición, p(1)
\begin{align*}
    p(1): &F_1 = \frac{L_{1-1} + L_{1+1}}{5} \\
    &F_1 = \frac{L_{0} + L_{2}}{5} \\
    &F_1 = \frac{2+3}{5} \\
    &F_1 = 1 \\
\end{align*}
Por enunciado, $ F_1 = 1 $ luego $p(1)$ es verdadero.

p(2)
\begin{align*}
    p(2): &F_2 = \frac{L_{2-1} + L_{2+1}}{5} \\
    &F_2 = \frac{L_{1} + L_{3}}{5} \\
    &F_2 = \frac{1+4}{5} \\
    &F_2 = 1 \\
\end{align*}

Por enunciado, $ F_2 = F_0 + F_1 = 1 $, luego $ p(2) $ es verdadero.

\textbf{Paso inductivo}

Busco probar que dado $h \geq 1: (p(h) \wedge p(h+1)) \implies p(h+2) $

HI: \\ 
$ F_h = \frac{L_{h-1} + L_{h+1}}{5} $ \\
$ F_{h+1} = \frac{L_{h} + L_{h+2}}{5} $ 

Qpq: $ F_{h+2} = \frac{L_{h+1} + L_{h+3}}{5} $

Pero,
\begin{align*}
    F_{h+2} &= F_h + F_{h+1} \\
    &= \frac{L_{h-1} + L_{h+1}}{5} + \frac{L_{h} + L_{h+2}}{5} \\
    &= \frac{L_{h-1} + L_{h+1} + L_{h} + L_{h+2}}{5} \\
    &= \frac{L_{h+1} + L_{h+3}}{5} \\
\end{align*}
Como se quería probar, luego vale el paso inductivo.

Así, $ p(n) $ es verdadero, $ \forall n \in \mathbb{N}_{\geq 1} $

\subsection{Ejercicio 3}

Busco determinar todos los $ (a,b) \in \mathbb{N}^2 $ tales que:
\begin{align*}
    (a:b) = -2a+b \wedge [a:b] = 83a
\end{align*}

Rdo.: $ (a:b)[a:b] = ab $

Usnado esto,
\begin{align*}
    (a:b)[a:b] = ab &\iff (-2a+b)(83a) = ab \\
    &\iff (-2a+b)83 = b \\
    &\iff -166a + 83b = b \\
    &\iff -166a + 82b = 0 \\
\end{align*}
Entonces, busco los $ (a,b) \in \mathbb{N}^2 $ que cumplen la ecuación diofácntica $ -166a + 82b = 0 $

\textbf{Verifico que existe solución}

Existe solución, pues $ MCD(166, 82) = 2|0 $

\textbf{Coprimizo la ecuación}

$ -166a+82b = -83a + 41b = 0 $

\textbf{Armo el conjunto solución}

$ s = \{ (a,b) \in \mathbb{N}^2: a = 41k \wedge b = 83k \wedge k \in \mathbb{Z} \} $

Para los valores hallados de $a$ y $b$ busco aquellos que cumplen con el MCD y el MCM 

$ (a,b) = (41k, 83k) \implies \begin{cases}
    (a:b) = (41k:83k) = k(41:83) = k \\
    [a:b] = [41k:83k] = k[41:83] = k.41.83
\end{cases} $

Luego,
\begin{align*}
    -2a + b = k &\iff -2(41k) + 83k = k \\
    &\iff -82k + 83k = k \\
    &\iff k = k \\
\end{align*}
Y,
\begin{align*}
    83a = k.41.83 \iff a = 41k
\end{align*}
Por lo tanto $ \{ (a,b): a = 41k \wedge b = 83k \wedge k \in \mathbb{Z} \} $ son todos los que cumplen lo pedido

\subsection{Ejercicio 4}

Se que $ (x-a)^3 | f $ y que $ (x-a)^2 | f' \iff f'=(x-a)^2.q $ con $ q(a) \neq 0 $

Luego $ q = (x-a)k +r $ y por algoritmo de división se que $ r = 0 $ o $ gr(r) < 1 $

Por lo tanto,
\begin{align*}
    f' = (x-a)^2 \cdot \left[ (x-a)k + r \right]
    f' = (x-a)^3k + (x-a)^2r
\end{align*}
Luego $ (x-a)^2r $ es el resto de dividir a $f'$ por $(x-a)^3 $ y $ r \in \mathbb{C} $

\end{document}
