\documentclass{article}
\usepackage{ifthen}
\usepackage{amssymb}
\usepackage{multicol}
\usepackage{graphicx}
\usepackage[absolute]{textpos}
\usepackage{amsmath, amscd, amssymb, amsthm, latexsym}
\usepackage{xspace,rotating,dsfont,ifthen}
\usepackage[spanish,activeacute]{babel}
\usepackage[utf8]{inputenc}
\usepackage{pgfpages}
\usepackage{pgf,pgfarrows,pgfnodes,pgfautomata,pgfheaps,xspace,dsfont}
\usepackage{listings}
\usepackage{multicol}
\usepackage{todonotes}
\usepackage{url}
\usepackage{float}
\usepackage{framed,mdframed}
\usepackage{cancel}

\usepackage[strict]{changepage}


\makeatletter


\newcommand\hfrac[2]{\genfrac{}{}{0pt}{}{#1}{#2}} %\hfrac{}{} es un \frac sin la linea del medio

\newcommand\Wider[2][3em]{% \Wider[3em]{} reduce los m\'argenes
\makebox[\linewidth][c]{%
  \begin{minipage}{\dimexpr\textwidth+#1\relax}
  \raggedright#2
  \end{minipage}%
  }%
}


\@ifclassloaded{beamer}{%
  \newcommand{\tocarEspacios}{%
    \addtolength{\leftskip}{4em}%
    \addtolength{\parindent}{-3em}%
  }%
}
{%
  \usepackage[top=1cm,bottom=2cm,left=1cm,right=1cm]{geometry}%
  \usepackage{color}%
  \newcommand{\tocarEspacios}{%
    \addtolength{\leftskip}{3em}%
    \setlength{\parindent}{0em}%
  }%
}

\usepackage{caratula}
\usepackage{enumerate}
\usepackage{hyperref}
\usepackage{graphicx}
\usepackage{amsfonts}
\usepackage{enumitem}
\usepackage{amsmath}

\decimalpoint
\hypersetup{colorlinks=true, linkcolor=black, urlcolor=blue}
\setlength{\parindent}{0em}
\setlength{\parskip}{0.5em}
\setcounter{tocdepth}{3} % profundidad de indice
\setcounter{section}{0} % nro de section
\renewcommand{\thesubsubsection}{\thesubsection.\Alph{subsubsection}}
\graphicspath{ {images/} }

% End latex config

\begin{document}

\titulo{Final 11/06/2021}
\fecha{2do cuatrimestre 2021}
\materia{Álgebra I}
\integrante{Yago Pajariño}{546/21}{ypajarino@dc.uba.ar}

%Carátula
\maketitle
\newpage

%Indice
\tableofcontents
\newpage

% Aca empieza lo propio del documento
\section{Final 11/06/2021}

\subsection{Ejercicio 3}

Se que $ 442 = 2.13.17 $

Luego,
\begin{align*}
    (4n^{49} + n + 33:442) = 221 &\iff (4n^{49} + n + 33:2.13.17) = 13.17 \\
    &\iff \begin{cases}
        13 | 4n^{49} + n + 33 \\
        17 | 4n^{49} + n + 33 \\
        2 \not | 4n^{49} + n + 33 \\
    \end{cases} \\
\end{align*}
Ahora busco los $n$ que cumplan cada una de las ecuaciones.

\textbf{Caso 13}
\begin{align*}
    13 | 4n^{49} + n + 33 &\iff 4n^{49} + n + 33 \equiv 0 (13) \\
    &\iff 4n^{49} + n \equiv 6 (13) \\
\end{align*}
Ahora separo en casos: $ 13|n$ y $13 \not | n $

\textbf{Caso $13|n$ }
\begin{align*}
    13 | n \implies 4n^{49} + n \equiv 0 + 0 \not \equiv 6 (13)
\end{align*}
Luego $ n \not \equiv 0(13) $

\textbf{Caso $13\not |n$ }
\begin{align*}
    13 \not | n &\implies 4n^{49} + n \equiv 6 (13) \\
    &\iff 4n^{r_{12}(49)} + n \equiv 6 (13) \\
    &\iff 4n + n \equiv 6 (13) \\
    &\iff 5n \equiv 6 (13) \\
    &\iff (-5)5n \equiv (-5)6 (13) \\
    &\iff -25n \equiv -30 (13) \\
    &\iff n \equiv 9 (13) \\
\end{align*}
Luego $ n \equiv 9(13) $

\textbf{Caso 17}
\begin{align*}
    17 | 4n^{49} + n + 33 &\iff 4n^{49} + n + 33 \equiv 0 (17) \\
    &\iff 4n^{49} + n \equiv 1 (17) \\
\end{align*}
De nuevo tengo dos casos $ 17|n$ y $17 \not | n $

\textbf{Caso $17|n$}
\begin{align*}
    17 | n \implies 4n^{49} + n \equiv 1 (17) \iff 0 + 0 \equiv 1 (17)
\end{align*}
Luego $ n \not \equiv 0 (17) $

\textbf{Caso $17\not |n$}
\begin{align*}
    17 \not | n &\implies 4n^{49} + n \equiv 1 (17) \\
    &\iff 4n^{r_{16}(49)} + n \equiv 1 (17) \\
    &\iff 4n + n \equiv 1 (17) \\
    &\iff 7.5n \equiv 7.1 (17) \\
    &\iff 35n \equiv 7 (17) \\
    &\iff n \equiv 7 (17) \\
\end{align*}
Luego $ n \equiv 7(17) $

\textbf{Caso 2}
\begin{align*}
    2 \not | 4n^{49} + n + 33 &\iff 4n^{49} + n + 33 \not \equiv 0 (2) \\
    &\iff \begin{cases}
        n \equiv 0 (2) \implies 0 + 0 + 33 \equiv 1 \not \equiv 0 (2) \\
        n \equiv 1 (2) \implies 4 + 1 + 33 \equiv 0 (2)
    \end{cases}
\end{align*}
Luego $ n \equiv 0 (2) $

Entonces juntando todo lo hayado, $S = \begin{cases}
    n \equiv 9(13) \\
    n \equiv 3(17) \\
    n \equiv 0(2) \\
\end{cases} $

Por el Teorema Chino del Resto se que existe una única solución mod 442, que es lo que busco.

Separo $ S $ en tres sistemas:

$ S_0 = \begin{cases}
    n \equiv 9(13) \\
    n \equiv 0(17) \\
    n \equiv 0(2) \\
\end{cases} $
$ S_1 = \begin{cases}
    n \equiv 0(13) \\
    n \equiv 3(17) \\
    n \equiv 0(2) \\
\end{cases} $
$ S_2 = \begin{cases}
    n \equiv 0(13) \\
    n \equiv 0(17) \\
    n \equiv 0(2) \\
\end{cases} $

Busco soluciones a cada sistema.

$ S_0 = \begin{cases}
    n \equiv 9(13) \\
    n \equiv 0(34) \\
\end{cases} \implies n = 34k \implies 34k \equiv 9(13) \iff k \equiv 8 (13) $

Luego $ x_0 = 8.34 = 272 $

$ S_1 = \begin{cases}
    n \equiv 3(17) \\
    n \equiv 0(26) \\
\end{cases} \implies n = 26k \implies 26k \equiv 3(17) \iff k \equiv 6 (17) $

Luego $ x_1 = 6.26 = 156 $

$ S_2 = \begin{cases}
    n \equiv 0(2) \\
    n \equiv 0(17.13) \\
\end{cases} $

Luego $ x_2 = 0 $

Entonces $ x = x_0 + x_1 + x_2 = 272 + 156 + 0 = 428 $

Por lo tanto, $ n \equiv 428(442) \implies r_{442}(n) = 428 $

\subsection{Ejercicio 4}

Sea $ n \in \mathbb{N} $ fijo, se define $ R $ una relación en $ \mathbb{C} - \{ 0 \} $ tal que
\begin{equation}
    z R w \iff \text{ existe } \alpha \in G_n \text{ tal que } z = \alpha w
\end{equation}
\subsubsection{Pregunta i}
Probar que es de equivalencia. Voy a probar cada propiedad por separado.

\textbf{Reflexividad}

Por definición de reflexividad, $ R $ es reflexiva $ \iff \forall k \in \mathbb{C}-\{ 0 \}: kRk $

Por (1), $ kRk \iff \text{ existe } \alpha \in G_n \text{ tal que } k = \alpha k $

Pero $ \forall n \in \mathbb{N} $, $ 1 \in G_n $ pues $ (1)^n = 1 $, luego $ k = k $ y por lo tanto $ R $ es reflexiva.

\textbf{Simetría}

Por definición de simetría, $ R $ es simétrica $ \iff \forall (k, j) \in (\mathbb{C} - \{ 0 \})^2: kRj \implies jRk $

Por (1), $ kRj \iff \text{ existe } \alpha \in G_n \text{ tal que } k = \alpha j $

Y quiero probar $ jRk \iff \text{ existe } \alpha \in G_n \text{ tal que } j = \alpha k $

Por lo tanto,
\begin{align*}
    &k = \alpha j; \alpha \in G_n \\
    \implies & j = \beta \alpha j \iff \beta \alpha = 1
\end{align*}
Pero dado que $ \alpha \in G_n \iff \alpha^{-1} \in G_n $ y $ \alpha \cdot \alpha^{-1} = 1 $

Por lo tanto $ \beta = \alpha^{-1} \implies j = \alpha^{-1}k $

Luego R es simétrica

\textbf{Transitividad}

Por definición de transitividad, $R$ es transitiva $ \iff \forall (j,k,l) \in (\mathbb{C} - \{ 0 \})^3: (jRk \wedge kRl) \implies jRl $

Por (1), \\
$ jRk \iff \exists \alpha \in G_n: j = \alpha k $ \\
$ kRl \iff \exists \alpha \in G_n: k = \alpha l $ \\
$ jRl \iff \exists \alpha \in G_n: j = \alpha l $

Luego $ jk = \alpha k \beta l \iff j = \alpha \beta l $

Entonces queda demostrar que $ \alpha \beta \in G_n $, pero se que $ \alpha^n = 1 $ y $ \beta^n = 1 $

Por lo tanto, $ (\alpha \beta)^n = \alpha^n \beta^n = 1 \implies \alpha \beta \in G_n $

Luego R es transitiva.

\subsubsection{Pregunta ii}

$ z = 3+5i $ y $ n = 4 $

Busco el conjunto de los $ w \in \mathbb{C} - \{ 0 \}: zRw $

Por (1), $ zRw \iff \exists \alpha \in G_4 / 3+5i = \alpha \cdot w $

Pero $ \alpha \in G_4 \iff \alpha \in \{ \pm 1, \pm i \} $

Luego,
\begin{itemize}
    \item $ \alpha = 1 \implies w = 3+5i $
    \item $ \alpha = -1 \implies w = -3-5i $
    \item $ \alpha = i \implies w = 5-3i $
    \item $ \alpha = -i \implies w = -5+3i $
\end{itemize}
Luego $ \overline{3+5i} = \{ 3+5i, -3-5i, 5+3i, -5+3i \} $

\subsection{Ejercicio 5}

\subsubsection{Pregunta i}

Factorización sabiendo que una de las raíces es cúbica de la unidad.

$ \alpha \in G_3 \iff \alpha \in \{ 1, -\frac{1}{2} + \frac{\sqrt[]{3}}{2}i, -\frac{1}{2} - \frac{\sqrt[]{3}}{2}i \} $

Dado que $ P(1) \neq 0 \implies (x-(-\frac{1}{2} + \frac{\sqrt[]{3}}{2}i))(x-(-\frac{1}{2} - \frac{\sqrt[]{3}}{2}i))|P $

Luego, $ (x-(-\frac{1}{2} + \frac{\sqrt[]{3}}{2}i))(x-(-\frac{1}{2} - \frac{\sqrt[]{3}}{2}i))|P \iff (x^2 + x +1) |P $

Usando el algoritmo de división, $ f = (x^2 + x +1)(x^4 - 4x^2 + 4) $

Defino $ g = x^4 - 4x^2 + 4 $

Cambio de variable $ y = x^2 $

Luego $ g' = y^2 - 4y + 4 \implies g' = (y-2)^2$ 

Por lo tanto, $ (x-\sqrt[]{2})(x+\sqrt[]{2}) | g \iff (x^2 - 2) | g $

Usando el algoritmo de división, $ g = (x^2 - 2)(x^2 - 2) $

Por lo tanto, juntando todo lo encontrado.

\begin{itemize}
    \item $ f = (x-(-\frac{1}{2} + \frac{\sqrt[]{3}}{2}i))(x-(-\frac{1}{2} - \frac{\sqrt[]{3}}{2}i))(x-\sqrt[]{2})^2(x+\sqrt[]{2})^2 $ es la factorización en $ \mathbb{C}[x] $
    \item $ f = (x^2 + x +1)(x-\sqrt[]{2})^2(x+\sqrt[]{2})^2 $ es la factorización en $ \mathbb{R}[x] $
    \item $ f = (x^2 + x +1)(x^2 - 2)^2 $ es la factorización en $ \mathbb{Q}[x] $
\end{itemize}

\subsubsection{Pregunta ii}
\begin{itemize}
    \item $ mult(-\frac{1}{2} + \frac{\sqrt[]{3}}{2}i, f) = 1 $
    \item $ mult(-\frac{1}{2} - \frac{\sqrt[]{3}}{2}i, f) = 1 $
    \item $ mult(x+\sqrt[]{2}, f) = 2 $
    \item $ mult(x-\sqrt[]{2}, f) = 2 $
\end{itemize}

\end{document}
