\documentclass{article}
\usepackage{ifthen}
\usepackage{amssymb}
\usepackage{multicol}
\usepackage{graphicx}
\usepackage[absolute]{textpos}
\usepackage{amsmath, amscd, amssymb, amsthm, latexsym}
\usepackage{xspace,rotating,dsfont,ifthen}
\usepackage[spanish,activeacute]{babel}
\usepackage[utf8]{inputenc}
\usepackage{pgfpages}
\usepackage{pgf,pgfarrows,pgfnodes,pgfautomata,pgfheaps,xspace,dsfont}
\usepackage{listings}
\usepackage{multicol}
\usepackage{todonotes}
\usepackage{url}
\usepackage{float}
\usepackage{framed,mdframed}
\usepackage{cancel}

\usepackage[strict]{changepage}


\makeatletter


\newcommand\hfrac[2]{\genfrac{}{}{0pt}{}{#1}{#2}} %\hfrac{}{} es un \frac sin la linea del medio

\newcommand\Wider[2][3em]{% \Wider[3em]{} reduce los m\'argenes
\makebox[\linewidth][c]{%
  \begin{minipage}{\dimexpr\textwidth+#1\relax}
  \raggedright#2
  \end{minipage}%
  }%
}


\@ifclassloaded{beamer}{%
  \newcommand{\tocarEspacios}{%
    \addtolength{\leftskip}{4em}%
    \addtolength{\parindent}{-3em}%
  }%
}
{%
  \usepackage[top=1cm,bottom=2cm,left=1cm,right=1cm]{geometry}%
  \usepackage{color}%
  \newcommand{\tocarEspacios}{%
    \addtolength{\leftskip}{3em}%
    \setlength{\parindent}{0em}%
  }%
}

\usepackage{caratula}
\usepackage{enumerate}
\usepackage{hyperref}
\usepackage{graphicx}
\usepackage{amsfonts}
\usepackage{enumitem}
\usepackage{amsmath}

\decimalpoint
\hypersetup{colorlinks=true, linkcolor=black, urlcolor=blue}
\setlength{\parindent}{0em}
\setlength{\parskip}{0.5em}
\setcounter{tocdepth}{3} % profundidad de indice
\setcounter{section}{0} % nro de section
\renewcommand{\thesubsubsection}{\thesubsection.\Alph{subsubsection}}
\graphicspath{ {images/} }

% End latex config

\begin{document}

\titulo{Final 20/07/2021}
\fecha{2do cuatrimestre 2021}
\materia{Álgebra I}
\integrante{Yago Pajariño}{546/21}{ypajarino@dc.uba.ar}

%Carátula
\maketitle
\newpage

%Indice
\tableofcontents
\newpage

% Aca empieza lo propio del documento
\section{Final 20/07/2021}

\subsection{Ejercicio 1}

Busco funciones $ f: A = \{ 1,2,3,4,...,10,11 \} \rightarrow B = \{ 1,2,3,..,24 \} $ que cumplan:
\begin{itemize}
    \item (a) $f$ es inyectiva
    \item (b) $ 10 \leq f(2) \leq 20 $
    \item (c) $ f(3) + f(4) = 13 $
\end{itemize}

Pero por definición, $f$ es inyectiva $ \iff \forall (a,b) \in A: f(a) = f(b) \implies a = b $

Es decir, cada elemento de $ A $ solo puede estar asociado a 0 o 1 de $B$

Por (b) al elemento $ 2 $ solo le puedo asignar uno de $ \{ 10,11,...,20 \} $

Por (c), busco formas de sumar 13: $ 1+12; 2+11; 3+10; 4+9; 5+8; 6+7 $ y las permutaciones. Luego hay 12 formas de sumar 13.

Luego separo en dos casos: el primero el caso en el que $ f(3) $ o $ f(4) \in \{ 10,11,...,20 \} $ y el segundo $ f(3) $ o $ f(4) \not \in \{ 10,11,...,20 \} $

\textbf{Caso 1}

En este caso voy a tener $10$ posibilidades para asignarle a $2$ y $6$ para $ f(3); f(4) $

Quedarían $ 11-3 = 8 $ elementos a los que asignar $ 24-3 = 21 $ elementos.

Luego habrá $ 10.6.\frac{21!}{8!} $ funciones

\textbf{Caso 2}

En este caso hay $11$ posibilidades para asignar al $2$ y $6$ posibilidades para $ f(3); f(4) $

Luego habrá $ 11.6.\frac{21!}{8!} $ funciones

Por lo tanto en total habrá $ 10.6.\frac{21!}{8!} + 11.6.\frac{21!}{8!} = 21.6.\frac{21!}{8!} $ funciones.

\subsection{Ejercicio 2}

Busco $p$ primos tales que $ p^4 | 77^{p^2} + 91^{p-1} + 21! \cdot p \implies p | 77^{p^2} + 91^{p-1} + 21! \cdot p $

Se que $ p | 21! \cdot p \implies p | 77^{p^2} + 91^{p-1} $ y $ 77 = 11.7; 91 = 13.7 $

\textbf{Caso $ p = 7 $}

$ 77^{p^2} + 91^{p-1} \equiv 0 (7) $ 

$ p = 7 $ es posible solución.

\textbf{Caso $ p = 11 $}

$ 77^{p^2} + 91^{p-1} \equiv 3^{10} \equiv 1(11) $

$ p = 11 $ NO es solución.

\textbf{Caso $ p = 13 $}

$ 77^{p^2} + 91^{p-1} \equiv (-1)^{13^2} \equiv -1 (13) $

$ p = 13 $ NO es solución.

\textbf{Caso $ p \not \in \{ 7, 11, 13 \} $}
\begin{align*}
    77^{p^2} + 91^{p-1} &\equiv (77^p)^p + 91^{p-1} (p) \\
    &\equiv 77 + 1 (p) \\
    &\equiv 78 (p) \\
\end{align*}

Luego $ p |78 \iff p|2.3.13 $

$ p = 2 \implies 77^4 + 91 \equiv 1 + 1 \equiv 0 (2) $ \\
$ p = 3 \implies 77^9 + 91^2 \equiv 2^9 + 1 \equiv 0 (3) $

Luego $ p = 2 $ y $ p = 3 $ son posibles soluciones.

Por lo tanto, los posibles $p$ son $ 2, 3, 7 $

Ahora busco ver si para los $p$ hallados $ p^4 | 77^{p^2} + 91^{p-1} + 21! \cdot p $

Caso $ p = 2 $
\begin{align*}
    p = 2 \implies 16 | 77^4 + 91 + 21! .2 &\iff 77^4 + 91 + 21! .2 \equiv 0 (16) \\
    &\iff 13^4 + 11 + 0 \equiv 0 (16) \\
    &\iff 1 + 11 + 0 \equiv 0 (16) \\
    &\iff 12 \equiv 0 (16) \\
\end{align*}
Luego $ p = 2 $ NO es solución

Caso $ p = 3 $
\begin{align*}
    p = 3 \implies 81 | 77^9 + 91^2 + 21! .3 &\iff 77^9 + 91^2 + 21! .3 \equiv 0 (81) \\
    &\iff 11^9. 7^9 + 100 + 0 \equiv 0 (81) \\
    &\iff 26. 55 + 100 + 0 \equiv 0 (81) \\
    &\iff 72 \equiv 0 (81) \\
\end{align*}
Luego $ p = 3 $ NO es solución

Caso $ p = 7 $
\begin{align*}
    p = 7 \implies 7^4 | 77^{7^2} + 91^6 + 21! .7 &\iff 77^{7^2} + 91^6 + 21! .7 \equiv 0 (7^4) \\
    &\iff (7^7.11^7)^{7} + 7^6. 13^6 + 7.3.20....15.7.2....8.7....1 .7 \equiv 0 (7^4) \\
    &\iff 0 + 0 + 0 \equiv 0 (7^4) \\
    &\iff 0 \equiv 0 (7^4) \\
\end{align*}
Luego $ p = 7 $ es el uníco que cumple lo pedido.

\subsection{Ejercicio 3}

$ (4n^2 - 1:14) = 7 \implies \begin{cases}
    7 | 4n^2 - 1 \\
    2 \not | 4n^2 - 1 \\
\end{cases} $

Luego $ 4n^2 \equiv 1 (7) \iff n^2 \equiv 2(7) $
\begin{itemize}
    \item $ n \equiv 0(7) \implies n^2 \equiv 0 (7) $
    \item $ n \equiv 1(7) \implies n^2 \equiv 1 (7)$
    \item $ n \equiv 2(7) \implies n^2 \equiv 4 (7)$
    \item $ n \equiv 3(7) \implies n^2 \equiv 2 (7)$
    \item $ n \equiv 4(7) \implies n^2 \equiv 2 (7)$
    \item $ n \equiv 5(7) \implies n^2 \equiv 4 (7)$
    \item $ n \equiv 6(7) \implies n^2 \equiv 1 (7)$
\end{itemize}
Luego $ n \equiv 3(7) $ o $ n \equiv 4(7) $

Por otro lado $ 4n^2 - 1 \equiv 0 (2) \iff 4n^2 \equiv 1 (2) $

Pero esto no se cumple para ningún $n$, luego $ 2 \not | 4n^2 - 1; \forall n \in \mathbb{N} $

Luego busco los $n$ tales que $ n^n \equiv 3(7) $
\begin{itemize}
    \item $ n \equiv 3(7) \implies 3^n \equiv 3^{6k + r_6(n)} \equiv 3^{r_6(n)} \equiv 3(7) $
    \item $ n \equiv 4(7) \implies 4^n \equiv 4^{6k + r_6(n)} \equiv 4^{r_6(n)} \equiv 3(7) $
\end{itemize}
Luego
\begin{itemize}
    \item $ n \equiv 0(6) \implies 3^0 \equiv 1 (7) \wedge 4^0 \equiv 1 (7) $
    \item $ n \equiv 1(6) \implies 3^1 \equiv 3 (7) \wedge 4^1 \equiv 4 (7) $
    \item $ n \equiv 2(6) \implies 3^2 \equiv 2 (7) \wedge 4^2 \equiv 2 (7) $
    \item $ n \equiv 3(6) \implies 3^3 \equiv 6 (7) \wedge 4^3 \equiv 1 (7) $
    \item $ n \equiv 4(6) \implies 3^4 \equiv 4 (7) \wedge 4^4 \equiv 4 (7) $
    \item $ n \equiv 5(6) \implies 3^5 \equiv 5 (7) \wedge 4^5 \equiv 2 (7) $
\end{itemize}
Por lo tanto, $ n \equiv 1(6) \wedge n \equiv 3(7) $ es la unica solución.

Usando el teorema chino del resto existe una única solución mod $6.7 = 42 $ que cumple lo pedido.

A ojo veo que $ n \equiv 31(42) $ cumple lo pedido.

\end{document}
