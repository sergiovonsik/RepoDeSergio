\documentclass{article}
\usepackage{ifthen}
\usepackage{amssymb}
\usepackage{multicol}
\usepackage{graphicx}
\usepackage[absolute]{textpos}
\usepackage{amsmath, amscd, amssymb, amsthm, latexsym}
\usepackage{xspace,rotating,dsfont,ifthen}
\usepackage[spanish,activeacute]{babel}
\usepackage[utf8]{inputenc}
\usepackage{pgfpages}
\usepackage{pgf,pgfarrows,pgfnodes,pgfautomata,pgfheaps,xspace,dsfont}
\usepackage{listings}
\usepackage{multicol}
\usepackage{todonotes}
\usepackage{url}
\usepackage{float}
\usepackage{framed,mdframed}
\usepackage{cancel}

\usepackage[strict]{changepage}


\makeatletter


\newcommand\hfrac[2]{\genfrac{}{}{0pt}{}{#1}{#2}} %\hfrac{}{} es un \frac sin la linea del medio

\newcommand\Wider[2][3em]{% \Wider[3em]{} reduce los m\'argenes
\makebox[\linewidth][c]{%
  \begin{minipage}{\dimexpr\textwidth+#1\relax}
  \raggedright#2
  \end{minipage}%
  }%
}


\@ifclassloaded{beamer}{%
  \newcommand{\tocarEspacios}{%
    \addtolength{\leftskip}{4em}%
    \addtolength{\parindent}{-3em}%
  }%
}
{%
  \usepackage[top=1cm,bottom=2cm,left=1cm,right=1cm]{geometry}%
  \usepackage{color}%
  \newcommand{\tocarEspacios}{%
    \addtolength{\leftskip}{3em}%
    \setlength{\parindent}{0em}%
  }%
}

\usepackage{caratula}
\usepackage{enumerate}
\usepackage{hyperref}
\usepackage{graphicx}
\usepackage{amsfonts}
\usepackage{enumitem}
\usepackage{amsmath}

\decimalpoint
\hypersetup{colorlinks=true, linkcolor=black, urlcolor=blue}
\setlength{\parindent}{0em}
\setlength{\parskip}{0.5em}
\setcounter{tocdepth}{2} % profundidad de indice
\setcounter{section}{2} % nro de section
\renewcommand{\thesubsubsection}{\thesubsection.\Alph{subsubsection}}
\graphicspath{ {images/} }

% End latex config

\begin{document}

\titulo{Práctica 3}
\fecha{2do cuatrimestre 2021}
\materia{Álgebra I}
\integrante{Yago Pajariño}{546/21}{ypajarino@dc.uba.ar}

%Carátula
\maketitle
\newpage

%Indice
\tableofcontents
\newpage

% Aca empieza lo propio del documento
\section{Práctica 3}

\subsection{Ejercicio 1}

Por enunciado, $ A= \{ n \in V: n \geq 132 \} $

Y también, $ A^c = \{ n \in V: n < 132 \} $

Se que dado un elemento cualquiera, $ x \in V \iff (x \in \mathbb{N} \wedge x \bmod 15 = 0)$

Por lo tanto, $ A^c = \{ n \in V: (n < 132 \wedge n \bmod 15 = 0) \} $

Así, $ \#A^c = \lfloor \frac{132}{15} \rfloor = 8 $

Por extensión, $ A^c = \{ 15,30,45,60,75,90,105,120 \} $

\subsection{Ejercicio 2}

Defino el conjunto universal $ V = \{ n \in \mathbb{N}: n \leq 1000 \} $

Defino el conjunto $ T = \{ n \in \mathbb{N}: n \bmod 3 = 0 \} $

Defino el conjunto $ C = \{ n \in \mathbb{N}: n \bmod 5 = 0 \} $

Se que un número no es múltiplo de 3 si no pertenece a $T$ y no es multiplo de 5 si no pertenece a $C$

Luego $ \#T = \lfloor \frac{1000}{3} \rfloor = 333 $ y $ \#C = \lfloor \frac{1000}{5} \rfloor = 200 $

Pero existen números que son múltiplos de 3 y 5 a la vez. Los múltiplos de 15.

Sea $ Q = \{ n \in \mathbb{N}: n \bmod 15 = 0 \} $

Luego, $\#Q = \lfloor \frac{1000}{15} \rfloor = 66 $

Por lo tanto la cantidad de números menores a 1000 que no son multiplos ni de 3 ni de 5 son:

\begin{align*}
    & res = \#V - \#T - \#C + \#Q \\
    & res = 1000 - 333 - 200 + 66 \\
    & res = 533
\end{align*}

Luego existen 533 números naturales menores a 1000 que no son múltiplos de 3 ni de 5.

\subsection{Ejercicio 3}

$ \#(A \cup B \cup C) = \#A + \#B+ \#C - \#(A \cap B) - \#(A \cap C) - \#(B \cap C) + \#(A \cap B \cap C) $

\subsection{Ejercicio 4}
\subsubsection{Pregunta i}

Datos del enunciado:
\begin{enumerate}
    \item $ \#V = 150 $
    \item $ \#A = 83 $
    \item $ \#B = 67 $
    \item $ \#(A \cap B) = 45 $
\end{enumerate}

Luego,
\begin{align*}
    \#(A \cup B)^c &= \#V - \#(A\cup B) \\
    &= \#V - (\#A + \#B - \#(A \cap B)) \\
    &= 150 - (83 + 67 - 45) \\
    &= 45 \\
\end{align*}

\subsubsection{Pregunta ii}

Total de elementos en A = elementos sólo en A + elementos en la intersección A y B + elementos en la intersección A y C + elementos en la intersección A, B y C
63 = 30 + x + z + 7

Total de elementos en B = elementos sólo en B + elementos en la intersección A y B + elementos en la intersección B y C + elementos en la intersección A, B y C
30 = 13 + x + y + 7

Total de elementos en C = elementos sólo en C + elementos en la intersección A y C + elementos en la intersección B y C + elementos en la intersección A, B y C
50 = 25 + y + z + 7

Resolviendo las ecuaciones:

Desde la ecuación 2, despejamos y en función de x:

y = 10 - x

Desde la ecuación 3, despejamos z en función de y:

z = 18 - y
z = 18 - (10 - x)
z = 8 + x

Luego, reemplazamos los valores de y y z en la ecuación 1:

63 = 30 + x + (8 + x) + 7

Resolviendo para x:

x = 9

Luego, reemplazando en las ecuaciones 2 y 3:

y = 10 - 9 = 1
z = 8 + 9 = 17

Por lo tanto, el número de elementos en la intersección A y B es 9, en la intersección A y C es 17, en la intersección B y C es 1, y en la intersección A, B y C es 7.

En resumen:
\begin{enumerate}
    \item Cuántos alumnos estudian exactamente dos idiomas? $9+17+1=27$
    \item ¿Cuántos inglés y alemán pero no francés? $9$
    \item ¿Cuántos no estudian ninguno de esos idiomas? $110-102 = 8$
\end{enumerate}

Resoución de \href{https://github.com/E-Liq}{E-Liq}

\subsection{Ejercicio 5}

Datos del enunciado:
\begin{enumerate}
    \item Rutas BSAS - Ros = 3
    \item Rutas Ros - SF = 4
    \item Rutas SF - Req = 4
\end{enumerate}

Por lo tanto hay $ 3 \cdot 4 \cdot 2 = 24 $ formas de ir de Buenos Aires a Reconquista pasando por Rosario y Santa Fe.

\subsection{Ejercicio 6}
\subsubsection{Pregunta i}
Hay $ 8 \cdot 9\cdot 9\cdot 9 = 5832 $ números.

\subsubsection{Pregunta ii}
Calculando por el complemento:

Hay $ 9 \cdot 10\cdot 10\cdot 10 = 9000 $ números de cuatro cifras.

En el inciso anterior se calculó la cantidad de números que no tienen cierto dígito (calculado por 5, vale para 7).

Luego habrá $ 9000 - 5832 = 3168 $ números.

\subsection{Ejercicio 7}
Puede distribuirlos en $ 3^{17} $ formas.

\subsection{Ejercicio 8}

Defino $ A = \{ materias \}$, se que $ \#A = 5 $

Luego las posibles elecciones están dadas por $ \#P(A) = 2^5 = 32 $

Si tiene que cursar al menos dos materias, no puede elegir las opciones de cursar ninguna materia o una sola materia.

Así tiene $ 32 - 5 - 1 = 26 $ formas de cursar al menos dos materias.

\subsection{Ejercicio 9}

Se que A es de la forma $ A = \{ a_1, a_2, ... , a_n \} $

$R$ es una relación en $ A \times A \iff R \subseteq A \times A $: si $R$ es un subconjunto del producto cartesiano $ A \times A $

Luego la cantidad de relaciones en A será: $ \# P(A \times A) = 2^{n^2}$

\begin{enumerate}
    \item Reflexivas: $ 2^{n^2-2} $
    \item Simétricas: $ 2^{\sum_{k =1}^{n}k} = 2^{\frac{n(n+1)}{2}} $
    \item Simétricas: $ 2^{\sum_{k =1}^{n-1}k} = 2^{\frac{n(n-1)}{2}} $
\end{enumerate}

\subsection{Ejercicio 10}
\begin{enumerate}
    \item $ \#\{ f \in F / \text{f es función}\} = 12^5 $
    \item $ \#\{ f \in F / 10 \not \in \text{Im(f)} \} = 11^5 $
    \item $ \#\{ f \in F / 10 \in \text{Im(f)} \} = 12^5 - 11^5 $
    \item $ \#\{ f \in F / f(1) \in \{ 2,4,6 \} \} = 3 \cdot 12^4 $
\end{enumerate}

\subsection{Ejercicio 11}

\begin{enumerate}
    \item $7! = 5040$ funciones.
    \item $3! \cdot 4! = 144$ funciones.
\end{enumerate}

\subsection{Ejercicio 12}
De cinco cifras usando los dígitos $\{ 1,2,3,4,5 \}: 5!$

De cinco cifras usando los dígitos $\{ 1,2,3,4,5,6,7 \}: \frac{5!}{2!}$

De cinco cifras usando los dígitos $\{ 1,2,3,4,5,6,7 \}$ sin 2 en las cententas: $\frac{7!}{2!} \cdot \frac{4}{5}$

\subsection{Ejercicio 13}
Rdo. funciones inyectivas: Una función $f: A\rightarrow B$ es inyectiva sii $ (x \in A) \wedge (y \in A) \wedge (x\neq y) \implies f(x) \neq f(y)$ 

\begin{enumerate}
    \item $\frac{10!}{(10-7)!} = \frac{10!}{3!}$
    \item Para $f(1)$ tengo 5 opciones. Al resto todas menos las que ya fueron asignadas (9,8,7,...) $ \implies 5 \cdot \frac{9!}{3!}$
\end{enumerate}

\subsection{Ejercicio 14}
Defino $A = \{ 1,2,3,4,5,6,7 \}$ y $B = \{ 1,2,3,4,5,6,7 \}$

Luego $\#A = \#B = 7$

$f: A \rightarrow B \text{ es viyectiva } \iff \forall x \in A; \exists ! y \in B: f(x) = y$ \\
Y además me piden que $ f(\{ 1,2,3 \}) \subseteq \{ 3,4,5,6,7 \} $

Luego habrá $ \frac{5!}{2!} \cdot 4! $ funciones que cumplen lo pedido.

\subsection{Ejercicio 15}
Tengo $R$ relación de equivalencia en $A=\{ f: \{ 1,2,3,4 \} \rightarrow \{ 1,2,3,4,5,6,7,8 \}: \text{f es inyectiva} \}$

Por definición, $fRg \iff f(1) + f(2) = g(1) + g(2)$

Necesito saber cuantas $g \in A$ se relaciones con $ f(n) = n+2 $

Pero,
\begin{align*}
    fRg \iff f(1) + f(2) &= g(1) + g(2) \\
    3+4 &= g(1) + g(2) \\
    7 &= g(1) + g(2) \\
\end{align*}

Entonces, busco las $g \in A: g(1) + g(2) = 7$ 

Hay seis funciones de $\{ 1,2 \} \rightarrow \{ 2,3,4,5,6 \}$ que cumplen con esto.

Completo el total de funciones asignando el resto de los elementos de forma inyectiva.

Luego habrá $ 6 \cdot \frac{6!}{4!} = 180$ elementos dentro de la clase de equivalencia de $f(n) = n+2$

\subsection{Ejercicio 16}

Defino $A=\{ 1,2,3,...8 \}$ y $B=\{ 1,2,3,...,12 \}$ con $\#A = 8$ y $ \#B = 12 $

Condiciones que me piden:
\begin{enumerate}
    \item f inyectiva
    \item $f(5) + f(5) = 6$
    \item $ f(1) \leq 6 $
\end{enumerate}

Primero busco asignaciones a $f(5)$ y $f(6)$ que cumplan lo pedido. Para esto hay cuatro opciones posibles.

Luego $f(1)$ puede tomar cualquier valor menos los dos que ya fueron asignados ya que $f(5); f(6)$ siempre toman valores $\leq 6$. Luego para $f(1)$ hay 4 opciones.

Para los demás elementos de $A$ pueden tomar alguno de los 9 elementos restantes de $B$.

Por lo tanto hay $4 \cdot 4 \cdot \frac{9!}{4!} $ opciones.

\subsection{Ejercicio 17}
\begin{enumerate}
    \item $\binom{7}{4}$
    \item $\binom{6}{3}$
    \item $\binom{6}{4}$
    \item $\binom{5}{3} \cdot 2$
\end{enumerate}

\subsection{Ejercicio 18}
Por enunciado $A = \{ n \in \mathbb{N}: n \leq 20 \}$ y $\#A = 20$

\subsubsection{Pregunta i}
Defino $B_1 = \{ n \in \mathbb{N}: n \leq 20 \wedge n \bmod 3 = 0 \} = \{ 3,6,9,12,15,18 \}$

Luego para armar las funciones debo elegir 4 del conjunto $B_1$ y 6 elementos del conjunto $B - B_1$

Luego habrá $ \binom{6}{4} \cdot \binom{14}{6} $ subconjuntos.

\subsubsection{Pregunta ii}
Hay suma impar de dos elementos si uno de ellos es par y el otro impar. Entonces, todos los elementos deben ser pares o impares.

Si son todos pares $ \implies \binom{10}{5} $ subconjuntos.

Si son todos impares $ \implies \binom{10}{5} $ subconjuntos.

Luego habrá $ 2 \cdot \binom{10}{5} $

\subsection{Ejercicio 19}
Cada punto de una recta se une a dos de la otra para formar un triángulo. 

Es decir, para cada vértice en una recta, elijo dos en la otra recta para formar el triángulo.

Luego habrá $ \binom{m}{2} \cdot n $ con $ m \geq 2; n\in \mathbb{N} $

\subsection{Ejercicio 20}
Defino $ A = \{ 1,2,3,...,11 \} $ y $ B = \{ 1,2,3,...,16 \} $

Me piden:
\begin{enumerate}
    \item f inyectiva
    \item $ n, f(n) $ pares
    \item $ f(1) < f(3) < f(5) < f(7) $
\end{enumerate}

La segunda condición me dice que los pares solo pueden tener imagen par, luego habrá $ \#fp $ funciones para los pares.

$ \#fp = \frac{8!}{3!} $

Para los impares tengo que considerar la tercera condición, esta implica que no me importa el orden de los elementos de B, sino que me voy a quedar con aquel que cumple la condición.

Así habrá $\#fi$ funciones para los impares.

$\#fi = \binom{11}{4} \cdot 7 \cdot 6$

Por lo tanto, hay $ \frac{8!}{3!} \cdot \binom{11}{4} \cdot 7 \cdot 6 $ funciones que cumplen lo pedido.

\subsection{Ejercicio 21}
\begin{enumerate}
    \item $ 7! $
    \item $ \frac{7!}{3!} $
    \item $ \frac{12!}{3!\cdot 2!} $
\end{enumerate}

\subsection{Ejercicio 22}
\begin{enumerate}
    \item $ \binom{7}{3} \cdot 3! \cdot 4! $
    \item $ \binom{7}{4} \cdot 3! $
    \item $ 4! \cdot 4! $
\end{enumerate}

\subsection{Ejercicio 23}
\begin{enumerate}
    \item Por el complemento: $ \frac{10!}{3! . 2!} - \frac{9!}{3!}$
    \item $ \binom{10}{3} \cdot 3! \cdot 7!$
\end{enumerate}

\subsection{Ejercicio 24}
Defino $ F = \{ D,D,D,D,D,D,N,N,B,P,H,K,C,M \} $

Condiciones:
\begin{enumerate}
    \item Dos frutas por día.
    \item No más de una N por día.
\end{enumerate}

Calculo por el complemento,

$ \# \text{Todas} - \# \text{Dos naranjas por día} = 14! - 7 \cdot 12!$

\subsection{Ejercicio 25}

Hay 15 personas pero A Juan y Nicolás los puedo pensar como bloque (JN), luego tengo 14 elementos para ordenar.

Calculo por el complemento: 
\begin{align*}
    Rta. &= \# \text{Todas las formas donde JN va en auto} - \# \text{LMD no van en auto y JN va en auto} \\
    &= 3 \cdot \binom{13}{2} \cdot \binom{11}{4} \cdot \binom{7}{4} - 3 \cdot \binom{10}{2} \cdot \binom{8}{4} \cdot \binom{4}{4}
\end{align*}

\subsection{Ejercicio 26}
Hago la demostración por inducción.

Defino $ p(n): \binom{2n}{n} > n\cdot 2^n; \forall n \in \mathbb{N}_{\geq 4} $

\textbf{Caso base n=4}

$ p(4): \binom{8}{4} > 4 \cdot 2^4 \iff \frac{8!}{4! \cdot 4!} > 4 \cdot 2^4 \iff 70 > 64 $

Luego p(4) es verdadero.

\textbf{Paso inductivo}

Dado $ k \geq 4 $ quiero probar que $ p(k) \implies p(k+1) $

HI: $ \binom{2k}{k} > k \cdot 2^k $

Qpq: $\binom{2(k+1)}{k+1} > (k+1) \cdot 2^{k+1} \iff \binom{2k+2}{k+1} > (k+1) \cdot 2^{k+1} $

Pero,
\begin{align*}
    \binom{2k+2}{k+1} &= \frac{(2k+2)!}{(k+1)! \cdot (2k+2-h-1)!} \\
    &= \frac{(2k+2) \cdot (2k+1) \cdot (2k)!}{(k+1)\cdot k! \cdot (k+1) \cdot k!} \\
    &> \frac{(2k+2)(2k+1)(k.2^k)}{(k+1)^2}
\end{align*}

Luego alcanza probar que,
\begin{align*}
    \frac{(2k+2)(2k+1)(k.2^k)}{(k+1)^2} &\geq (k+1)\cdot 2^{k+1} \\
    \frac{2(k+1)(2k+1)(k.2^k)}{(k+1)(k+1)} &\geq (k+1)\cdot 2^{k} \cdot 2 \\
    \frac{(2k+1)\cdot k}{k+1} &\geq k+1 \\
    2k^2 + k &\geq k^2 + 2k +1 \\
    k^2 - k &\geq 1 \\
    k \cdot (k-1) &\geq 1 \\
\end{align*}

Que es verdadero, $ \forall k \geq 4 $.

Luego $ p(k) \implies p(k+1) $ como se quería probar.

Así, $ p(n) $ es verdadero, $\forall n \in \mathbb{N}_{\geq 4} $

\subsection{Ejercicio 27}
Lo pruebo por inducción.

Defino $ p(n): a_n = \binom{2n}{n} \forall n \in \mathbb{N} $

\textbf{Caso base n=1}

$ p(1): a_1 = \binom{2.1}{1} = 2$

Por definición de la sucesión, $ a_1 = 2 $

Luego $ p(n) $ es verdadero.

\textbf{Paso inductivo}

Dado $ k \geq 1 $ quiero probar que $ p(k) \implies p(k+1) $

HI: $ a_k = \binom{2k}{k} $

Qpq: $a_{k+1} = \binom{2(k+1)}{k+1} = \binom{2k+2)}{k+1} $

Pero,
\begin{align*}
    a_{k+1} &= 4 \cdot a_k - 2 \cdot \frac{(2k)!}{(k+1)! \cdot k!} \\
    &= 4 \cdot \binom{2k}{k} - 2 \cdot \frac{(2k)!}{(k+1)! \cdot k!} \\
    &= 4 \cdot \frac{(2k)!}{k! \cdot k!} - 2 \cdot \frac{(2k)!}{(k+1)! \cdot k!} \\
    &= \frac{4\cdot (k+1)\cdot (2k)! - 2 \cdot (2k)!}{(k+1)! \cdot k!} \\
\end{align*}

Luego alcanza probar que,
\begin{align*}
    \frac{4\cdot (k+1)\cdot (2k)! - 2 \cdot (2k)!}{(k+1)! \cdot k!} &= \binom{2k+2)}{k+1} \\
    \frac{4\cdot (k+1)\cdot (2k)! - 2 \cdot (2k)!}{(k+1)! \cdot k!} &= \frac{(2k+2)!}{(k+1)!(k+1)!} \\
    \frac{4\cdot (k+1) - 2}{k!} &= \frac{(2k+2) \cdot (2k+1)}{(k+1)!} \\
    4\cdot (k+1) - 2 &= \frac{2(k+1)(2k+1)}{k+1} \\
    4k + 4 - 2 &= 4k + 2 \\
    4k + 2 &= 4k + 2 \\
\end{align*}

Luego $ p(k) \implies p(k+1) $ como se quería probar.

Así, $ p(n) $ es verdadero, $\forall n \in \mathbb{N} $

\subsection{Ejercicio 28}
TODO

\subsection{Ejercicio 29}

Enunciado, $ X = \{ 1,2,3,...,20 \} $ y $R$ una relación en $P(X)$

Por definición, sean $A \in P(X); B \in P(X)$ conjuntos, $ ARB \iff A - B = \emptyset \iff A \subseteq B $

Luego busco $ A \in P(X): (\#A \geq 2) \wedge (AR\{ 1,2,3,4,5,6,7,8,9 \})$

Por el complemento: $ \# P(\{ 1,2,...,9 \}) - \# \{ c: \# c < 2 \wedge c-\{ 1,2,...,9 \} = \emptyset \} $

Luego habrá $ 2^9 - \left[ \binom{9}{0} + \binom{9}{1} \right] = 2^9 - 10 $ subconjuntos.

\subsection{Ejercicio 30}
Por enunciado, $ X = \{ 1,2,3,4,5,5,7,8,9,10 \} $

Por definición, $ ARB \iff A \cap \{ 1,2,3 \} = B\cap \{ 1,2,3 \} $

Luego busco conjuntos $ B \in P(X): (\#B = 5) \wedge (BR\{ 1,3,5 \})$

Pero $ BR\{ 1,3,5 \} \iff B\cap \{ 1,2,3 \} = \{ 1,3,5 \} \cap \{ 1,2,3 \} = \{ 1,3 \} $

Así, busco subconjuntos de X de 5 elementos que incluyan al $\{ 1,3 \}$ y no tengan al 2.

Entonces, hay $ \binom{7}{3} = 35 $ subconjuntos.

\subsection{Ejercicio 31}
Por enunciado, $ X = \{ n \in \mathbb{N}: n \leq 100 \} $ y $ A = \{ 1 \} $

Se que $ A \triangle B \iff (A \cup B) - (A \cap B) $

Entonces, busco $B$ tales que $ A \triangle B $ tengan 0 o 1 o 2 elementos.

Para obtener 0 elementos $ B = \{ 1 \} \implies A \triangle B = (A \cup B) - (A \cap B) = \{ 1 \} - \{ 1 \} = \emptyset $

Hay un elemento.

Para obtener 1 elemento $ B = \{ b_1, 1 \} \implies A \triangle B = (A \cup B) - (A \cap B) = \{ b_1, 1 \} - \{ 1 \} = \{ b_1 \}$

Luego hay $ \binom{99}{1} + 99 $ que cumplen esto.

Para obtener 2 elementos $ B = \{ b_1, b_2, 1 \} \implies A \triangle B = (A \cup B) - (A \cap B) = \{ b_1, b_2, 1 \} - \{ 1 \} = \{ b_1, b_2 \}$

Luego hay $ \binom{99}{2} $ que cumplen esto.

Así, habrá $ 1 + \binom{99}{1} + 99 + \binom{99}{2} = 5050 $

\subsection{Ejercicio 32}
\subsubsection{Pregunta i}

Tengo un conjunto $A$ con n elementos. Busco que la relación de equivalencia de $ a \in A $ tenga n elementos.

La relación de equivalencia me dice con cuantos elementos se relaciona, en este caso, el elemento $a$.

Luego voy a tener tantas clases de equivalencia como formas de elegir $n$ elementos de un conjunto de $2n$ elementos.

Así, habrá $ \binom{2n}{n} $ clases de equivalencia.

\subsubsection{Pregunta ii}
Con el mismo rezanamiento que el inciso anterior habrá, $ \binom{3n}{b} \cdot \binom{2n}{n} $ clases de equivalencia. 

\end{document}
